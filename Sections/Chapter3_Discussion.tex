\section{Discussion}

The results of our study support two novel insights regarding the function of higher-order motor cortex in adaptive choice behavior. Firstly, fast and slow ensemble transitions are neural signatures for distinct phases of voluntary behavior. A comparison between transitions was possible because our task design allowed for multiple shifts between multiple contingencies within a single behavioral session. Secondly, the relative timing of neural and behavioral shifts, as well as the specific deficits following inactivation, highlighted a leading role for this region in the engagement of sensory cue-guided actions (Fig. \ref{fig:NN_figS9}). 

\input{Figures/NN_FigS9.tex}

This conclusion contrasts with previous studies of homologous or nearby prefrontal cortical regions, in which neural changes closely match or lag the time course of behavioral adaptation \citep{mitz1991learning,pasupathy2005different,durstewitz2010abrupt}. One explanation may be that prior studies have focused on the learning of novel sensorimotor mappings or new rules, whereas our task required animals to repeatedly disengage and re-engage the learned associations needed in sound-guided trials. 

Although this task shares important features with other assays for flexibility, there are also crucial differences. In contrast to paradigms that use a contextual cue to instruct rapid executive control on a trial-by-trial basis \citep{mante2013context,stokes2013dynamic,duan2015requirement}, animals in our experiment adapted on a time scale of tens of trials (Fig. 1c). This relatively slow rate of adaptation resembles that of learning during arbitrary visuomotor mapping, where the animal's basis for action selection is updated gradually based on reward feedback \citep{pasupathy2005different,asaad1998neural}. 

Our task also differs from other strategy- or set-shifting tasks for rodents \citep{durstewitz2010abrupt,darrah2008interaction}, in the sense that we used nonspatial stimuli that do not conform to classical definitions of exemplars or sets. Instead, the paradigm we used consisted of blocks of trials that required the subject to shift between conditional and non-conditional approaches to action selection. 

Analysis of the types of errors made during training suggests that mice perform two-choice auditory discrimination in part by suppressing a prepotent tendency to repeat a rewarded choice. Action trials could thus be considered a natural strategy to the animal, whereas sound-guided trials require weeks of training to achieve high performance. One caveat for our task is that animals are likely to have different degrees of learned and intrinsic familiarity for sound versus action trials. In principle, mice may solve the task by ignoring sensory information completely during action blocks. However, the temporally structured lick rates during action blocks (Fig. 1e) strongly suggest use of the stimulus for gating lick responses. 

We found that bilateral inactivation of M2 selectively impaired the shift into sound-guided actions. This observation is highly consistent with results of dorsal premotor lesions in primates, which disrupt both the learning of novel visuomotor associations and the engagement of previously learned mappings \citep{petrides1985deficits,halsband1985premotor,nixon2004cortico}. Adaptation to action blocks was facilitated by M2 inactivation. This effect could result from a tendency to repeat the prior choice \citep{sul2011role}: if M2 normally biases animals toward sensory-cue-guided actions, then inactivation may remove an important brake on the unconditional strategy. In our experiments, M2 inactivation slowed but did not preclude the eventual transition to high performance on sound-guided trials. This suggests that, at least for trained mice, two-choice auditory discrimination alone does not require M2 and may be subserved by other circuits \citep{znamenskiy2013corticostriatal}. Furthermore, we take the opposing effects of inactivation on shifts to sound-guided versus repeated actions as evidence that mice perform the task by balancing the use of conditional and unconditional responses.

A key finding of this study concerns how specific parameters of ensemble activity transitions may relate to behavior. We found that ensemble transitions were more abrupt when animals needed to retrieve and begin using conditional associations. These fast transitions may be related to those observed in medial prefrontal cortex, which have been interpreted as neural correlates of insight \citep{durstewitz2010abrupt} or as the abandonment of an inadequate internal model at the onset of exploration \citep{karlsson2012network}. 

On what quantitative basis should transitions be classified as abrupt or gradual? We compared the steepness of these transitions directly to the slower transitions that accompanied adaptation to action blocks. Moreover, ensemble transitions occurred after only a few errors, whereas behavioral improvements took tens of sound-guided trials (Fig. 4e,f). The difference in neural and behavioral timing suggests that M2 neural activity had mostly adjusted while the animal was still systematically responding in an unconditional manner. M2 may facilitate the engagement of sound-guided behavior by biasing the use of sensory information, suppressing repetitive actions, or both. By contrast, prior studies show that when an animal must acquire novel arbitrary associations, changes in cortical activity track behavioral improvements \citep{mitz1991learning,brasted2004comparison} and lag the more rapid remapping in the striatum \citep{pasupathy2005different}. A major difference between these studies of fast learning and our study is that the auditory–motor associations were already well learned in our task.

We found that multiple rules were each associated with a distinct subset of population activity patterns. Such task-dependent changes in neural activity are reported in multiple frontal cortical regions across species \citep{asaad2000task,rich2009rat,rodgers2014neural,durstewitz2010abrupt,wallis2001single}. By asking the animal to shift repeatedly during a single session, we found that the network could return to a previously employed functional configuration to meet similar behavioral demands. This back-and-forth toggling of ensemble activity is reminiscent of the ensemble remapping observed in CA1 of the hippocampus during repeated exposure to spatial contexts \citep{wills2005attractor,leutgeb2005independent}. One study reported that changes in environmental context also cause network activity shifts in the rodent medial prefrontal cortex. However, the ensemble code was not identical upon re-exposure, potentially owing to a systematic drift over time \citep{hyman2012contextual}. The divergent findings of repeatable versus drifting network states in the rodent frontal cortex could reflect regional differences or differences in how frontal areas encode cognitive versus environmental variables.

Several lines of evidence support the idea that the neural dynamics in M2 reflect changes in internal processes (for example, representation of task contingencies or motor planning and preparation) rather than differences in overt physical movements. Firstly, three different ensemble analyses with matched, congruent trial conditions indicated distinct neural dynamics in sound and action blocks (Fig. 5f–i and Supplementary Fig. 7), despite a lack of observable difference in motor output for the same sets of trials (Fig. 1e and Supplementary Fig. 2). Secondly, neural signals related to motor execution should be strongest at the time of response. Instead, we found that the rule-specific separation of population activity patterns was substantially above chance at all times across a trial (Fig. 5d). Perhaps the strongest evidence is that muscimol inactivation of M2 had no detectable effect on motor output (Supplementary Fig. 5), while clearly affecting behavioral flexibility.

What is the purpose of functional reconfiguration during adaptive decision-making? Ensemble activity patterns within multiple network subspaces reflect the diversity of neural representations in M2. Recent studies indicate functional roles for long-range projections from rodent M2 to sensory cortices \citep{schneider2014synaptic,manita2015top} and dorsal striatum \citep{rothwell2015input}. Appropriate shifts in neural representations could allow M2 to exert differential top-down control in a task-dependent manner. Further study regarding the downstream impacts of frontal network transitions may yield insights into neuropsychiatric disorders in which cognitive flexibility is impaired. Plausibly, the cognitive rigidity characteristic of disorders such as schizophrenia could result from an inability of frontal cortical networks to shift or maintain stable ensemble states.