\nnsec{Introduction}

Associations between past choices and their outcomes allow for efficient selection of actions likely to meet one’s present goals. The mechanisms through which such associations are implemented remain unclear. However, at the behavioral level, it is well known that rewarded choices tend to be repeated at the expense of those that have yielded meager or aversive results. To gain insight into the associative mechanisms underlying goal-directed action selection, the present study focuses on the question of how populations of simultaneously recorded neurons in the cerebral cortex represent and integrate information related to choices and their outcomes.

How does the brain selectively reinforce rewarded actions in order to bias their future implementation? Physiological studies in primates and rodents suggest that the frontal lobe plays an important role in these functions. For example, the primate prefrontal cortex is known to contain neurons that encode chosen actions and outcomes \citep{barraclough2004prefrontal,genovesio2006representation,seo2007dynamic,histed2009learning}, suggesting a plausible neural substrate for their association during goal-directed behavior. Moreover, single-unit recordings have revealed that prior reward enhances the discriminability of spiking activity related to past \citep{donahue2013cortical} and upcoming choices \citep{histed2009learning}.

Similarly, recordings from the medial frontal cortex (MFC) of rodents have revealed neural signatures of prior choices and outcomes \citep{sul2010distinct, sul2011role, hyman2017novel}. In particular, these studies have demonstrated neural representations of choice and outcome history in the most dorsal anatomical sub-region of MFC, which is referred to as secondary motor cortex (M2) in mice, and medial agranular or medial precentral cortex in rats \citep{sesack1989topographical, barthas2017secondary}. Murine M2 has also been implicated in the flexible acquisition and initiation of voluntary actions \citep{ostlund2009evidence, gremel2013premotor, murakami2014neural, siniscalchi2016fast, barthas2017secondary, makino2017transformation}. Based on its putative role in instrumental behavior, M2 may serve as an important interface for the mixing of choice- and reward-related signals in the rodent brain. However, the details of how reinforcement might interact with choice-related neural representations remain unclear. 

One intriguing hypothesis is that a choice’s outcome could affect the strength or persistence of its neural signature in M2. To explore this possibility, we trained mice on a two-choice auditory discrimination task and then introduced a probabilistic reinforcement schedule during testing. Simultaneous two-photon calcium imaging enabled the characterization of task-related neural ensemble activity in M2. Three randomly interleaved outcomes (single, double and omitted rewards) delivered following correct responses allowed us to measure the impact of reward on choice coding, as well as to distinguish effects of changes in reward magnitude from those of its absolute presence or absence. 

We found that rewarding outcomes boosted the fidelity of choice signals encoded in the population activity patterns—an effect that persisted into the subsequent trial. Importantly, the reward-dependent enhancement of choice-related signals depended less on differences in reward size than on the categorical presence or absence of rewards. These results suggest one plausible cortical mechanism for the reinforcement of rewarded actions—namely, that rewarding outcomes lead to a more robust population-level read-out of recent choice history.