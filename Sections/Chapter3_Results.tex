\section{Results}

\subsection{An adaptive decision-making task for head-fixed mice}

We trained head-fixed mice to perform a task requiring flexible sensorimotor mapping. In each trial, fluid-restricted mice were presented with one of two randomized auditory stimuli, either logarithmic frequency-modulated sweeps from 5 to 15 kHz (`upsweep') or from 15 to 5 kHz (`downsweep'), and had to respond with a lick to the left or right port (Fig. 1a and Supplementary Video 1). A correct response was rewarded with 2 \unit{\micro\liter} of water, while an incorrect response resulted in white noise. Trials were organized into blocks (Fig. 1b), each with a distinct set of stimulus–response contingencies: `sound-guided' (upsweep–left; downsweep–right), `action-left' (upsweep–left; downsweep–left), and `action-right' (upsweep–right; downsweep–right). When performance reached a criterion of 85\% correct over 20 trials, a new block began with different contingencies. Sound and action blocks alternated, and no contextual cue was given to signal the block transition. Therefore, performance beyond the first block required flexible response selection and outcome monitoring. Mice were prepared for this task by initial training to an expert level on two-choice auditory discrimination; i.e., ∼30 d on a task with only sound-guided trials. Here, we present data from mice with fewer than 6 sessions of experience in the adaptive decision-making task.

\begin{figure}[htbp]

\begin{center}
\includegraphics[width=\textwidth]{Figures/NN_fig1.jpg} 
\end{center}

\caption[Behavioral performance of head-fixed mice in an adaptive sensorimotor decision-making task]
{Behavioral performance of head-fixed mice in an adaptive sensorimotor decision-making task.
(A) Schematic of experiment. Each trial begins with an auditory cue. A response window starts 0.5 s after cue onset, during which the first lick is recorded as the response for that trial. Water reward is delivered contingent on a correct response. ITI, intertrial interval. (B) Schematic of stimulus–response mappings and block design. (C) Behavioral performance surrounding a block switch from action to sound (left) or sound to action (right). Filled circles, hit rate. Open circles, perseverative error rate. Dotted line, other error rate. $N=$ 33 action-to-sound switches and 38 sound-to-action switches. (D) Performance from one example behavioral session. Trial outcomes: correct (filled circles), perseverative error (open circles), other error (open triangles), or miss (cross). Vertical line, rule switch. (E) Left and right lick rates for upsweep or downsweep sound cues during all correct sound-guided (black), action-left (red) and action-right (blue) trials. For each choice direction, lick rates during action trials were compared to sound trials in 0.1 s bins; black bars, significant differences ($p < 0.01$, paired t-test). All data presented as $mean \pm SEM$. $N = 9$ sessions from 5 mice.}

\label{fig:NN_fig1}
\end{figure}

% \begin{FPfigure}

% \begin{center}
% \includegraphics[width=\textwidth]{Figures/CC_fig6.png} 
% \end{center}
% \small{Figure \ref{fig:CC_fig6} Decoding accuracy diminished during omitted-reward and error trials. }

As expected, a switch in contingencies was associated with an immediate drop in correct response rate (Fig. 1c,d). Most incorrect responses were perseverative errors, indicating a failure to update response strategy for ∼20 trials after the switch. We obtained concurrent calcium imaging and behavioral data during 9 sessions from 5 mice (Supplementary Table 1). On average, these mice performed 418 ± 49 trials per session, including 296 ± 38 rewarded trials and 9 ± 1 block switches (mean ± s.e.m.; range: 6–19 switches; Supplementary Fig. 1a). To quantify motor output, we calculated the mean lick rates and the time of first lick for different trial types. Overall, licks were tightly locked to the time of auditory cue during correct trials (Fig. 1e). For congruent trials (in which stimulus–response contingencies match), lick rates were indistinguishable across sound and action blocks. For incongruent trials (for example, left action for upsweep during sound block versus downsweep during action-left block), there was a noticeable difference in mean lick rates and an increased latency to first lick (Supplementary Fig. 2). Nevertheless, the major determinant for the shape of the lick distribution was response direction: i.e., whether the animal chose left or right (Fig. 1e). Additionally, we used video tracking to monitor whisker and hindpaw positions, and found that their movements also depended mostly on response direction (Supplementary Fig. 3). Therefore, although licks were the means for making operant responses in this head-fixed setup, mice performed more complex motor programs to indicate their choices.

\subsection{Silencing M2 selectively impairs shift to sound-guided actions}
To determine whether frontal cortical activity is necessary for adaptive decision-making in our task, we used the GABAA receptor agonist muscimol to inactivate M2 bilaterally. Muscimol (5 mM, 46 nL per hemisphere) or saline vehicle was injected ∼1 h before behavioral testing (n = 11 mice; Fig. 2a and Supplementary Table 1). We injected low-molecular-weight fluorescein to estimate the extent of the affected region, which included M2 and part of cingulate cortex (Cg1), but not other neighboring regions (Supplementary Fig. 4). Compared to controls, muscimol-injected mice performed fewer trials (Fig. 2b; saline: 608 ± 42, muscimol: 476 ± 31, mean ± s.e.m.; P = 0.007, W = 62, Wilcoxon signed-rank test), although there was no difference in the number of switches per 100 trials (saline: 2.7 ± 0.2, muscimol: 2.7 ± 0.1, mean ± s.e.m.; P = 0.96, W = 34). Notably, separate analyses of sound and action blocks revealed selective impairments in the animals' ability to engage sound-guided actions, evidenced by a marked (55\%) increase in the number of perseverative errors per block (Fig. 2c; saline: 5.7 ± 1.1, muscimol: 8.9 ± 1.9, mean ± s.e.m.; P = 0.042, W = 10, Wilcoxon signed-rank test), and a greater number of trials to reach criterion (saline: 38 ± 4, muscimol: 48 ± 6, mean ± s.e.m.; P = 0.042, W = 10). Nevertheless, muscimol-injected mice eventually reached the criterion of >85\% correct, indicating that the transition to a high level of performance was slowed, but not blocked, by M2 inactivation. Silencing had the opposite effect on shifts into action blocks, during which the mice required fewer trials to reach criterion, although this effect fell short of statistical significance (saline: 43 ± 4, muscimol: 32 ± 2, mean ± s.e.m.; P = 0.054, W = 55). Inactivation had no effect on the timing or rates of lick motor output (Supplementary Fig. 5). These results indicate a causal role for M2 in the flexible control of action selection. Additionally, the opposing effects of silencing are useful for understanding how the mouse performs the outlined task. One solution to the task would be to forget and relearn the relevant stimulus–response associations after each contingency change, as in a reversal task33. This approach predicts symmetric changes in behavior following perturbations. An alternative approach would be to rely on these associations for sound-guided trials and then ignore them during action blocks to favor repeated selection of the same response. In this case, the mouse would perform the task by shifting the balance between conditional and nonconditional means of responding. The asymmetric deficits observed in our experiments are consistent with the second approach and implicate M2 in the breaking of repetitive actions and biasing choices based on learned associations.

\subsection{Imaging task-related activity at cellular resolution in M2}
To characterize neural activity, we injected adeno-associated viruses encoding GCaMP6s into layer 2/3 of M2 (AAV1-Syn-GCaMP6s-WPRE-SV40; Fig. 3a). GCaMP6s is a genetically encoded calcium indicator that exhibits a ∼25\% rise in fluorescence intensity per action potential in cortical pyramidal neurons34. While mice performed the adaptive decision-making task, we used two-photon microscopy to record from 62 ± 6 cells per field of view (mean ± s.e.m.; range, 26–83 cells; n = 9 sessions from 5 mice; Fig. 3b). Figure 3c shows four example M2 neurons with fluorescence transients (ΔF/F) concurrent with responses during sound-guided trials. To examine how the use of conditional rules affects the activity of individual neurons, we averaged ΔF/F across correct trials for the congruent upsweep–left and downsweep–right conditions separately for sound and action blocks. Neural responses were diverse, even for neurons within the same field of view (Fig. 3d). During sound-guided trials, neurons could exhibit higher ΔF/F for specific associations—i.e., upsweep–left (cell 2) or downsweep–right (cells 1 and 3)—or have no preference (cell 4). The use of conditional rules modulated ΔF/F in some neurons (cells 1, 2 and 3) and in other cases had no effect (cell 4).

\subsection{Neural transition is more rapid during shift to sound rule}
The observed heterogeneity of neural responses opened the question of whether single-neuron activity in M2 reflects the components of an ensemble representation for specific task variables. If so, then population-level analyses might more effectively capture the content of such representations. Toward this end, we calculated population activity vectors from ΔF/F and used demixed principal component analysis (dPCA)35,36 to project the vectors in a reduced representational space (Online Methods). Plotting these vectors over time generates trajectories describing the time-dependent evolution of ensemble activity during behavior. To determine how the ensemble activity evolved around block switches on a trial-by-trial basis, we calculated the Mahalanobis distances between population activity vectors of each trial and those of the 20 trials before the last or next block switch. Following a contingency switch, we found that the ensemble activity migrated away from the previous representational subspace toward a new subspace associated with the new rule (Fig. 4a). Comparisons of the transition dynamics following a switch into conditional versus nonconditional rules uncovered marked differences. Out of 33 action-to-sound and 38 sound-to-action transitions, 33 and 35 switches, respectively, could be fit with a logistic function to compare the onset and rate of shifts in population activity patterns (Fig. 4b,c). State transitions associated with the shift to sound-guided responses occurred after only several trials, much earlier than with shifts into repeated actions (Fig. 4d,g; sound: midpoint trial (xo) = 4.0, action: xo = 10.4, median; P = 0.007, z = 2.70, Wilcoxon rank-sum test). Furthermore, breaking from repetitive to sound-guided responding involved transitions that were more abrupt (sound: steepness (k) = 1.02, action: k = 0.35, median; P = 0.03, z = –2.17, Wilcoxon rank-sum test). These differences in neural dynamics were not due to behavioral differences, because in this set of experiments trials to criterion were similar for the two rule types (sound: 39, action: 38, median; P = 0.9, z = 0.09, Wilcoxon rank-sum test; Supplementary Fig. 1a). Overall, these results suggest that ensemble activity patterns in M2 shift earlier and more steeply when animals are required to abort repetitive actions and engage conditional associations to perform sound-guided behavior.

To what extent must population activity resemble the final ensemble state in order to improve behavior? To address this question, we performed two analyses to compare the timing of neural and behavioral transitions. In the first analysis, we defined `transition trials' for behavior (trials to criterion minus 20, the sliding window for assessing criterion) and neural ensemble activity (Mahalanobis distance ratio equaling 75\% L based on logistic fit; Online Methods). Block-by-block paired comparisons of neural and behavioral transition trials showed that ensemble activity in M2 shifted before the recovery of behavioral performance when adapting to conditional rules (Fig. 4e and Supplementary Fig. 6; P = 0.003, z = –2.96; Wilcoxon signed-rank test). By contrast, neural and behavioral changes occurred at around the same time for shifts to nonconditional responding (Fig. 4h; P = 0.19, z = 1.32; Wilcoxon signed-rank test). We should note, however, that the definitions used for transition trials were arbitrary. Therefore, we performed a second, less biased analysis in which we determined the mean performance at the behavioral trial corresponding to a series of different neural transition locations. Compared with shifts to action trials (Fig. 4i), transitions to sound-guided trials were associated with hit and error rates that diverged later (Fig. 4f), indicating that behavioral improvement occurred later along the time course of neural transitions. Taken together, these two analyses suggest that when shifting to sound-guided actions, neural ensemble transitions in M2 are nearly complete before behavioral performance improvement can be detected.

\subsection{Distinct activity patterns accompany rule implementations}
Our results indicated that rule shifts are associated with distinct transitions in network activity. This leads naturally to the question of what ensemble dynamics accompany successful rule implementation. We examined trajectories associated with correct responses in the 20 trials before the switch, when response strategies had stabilized (>85\% correct by task design). Figure 5a shows the trajectories of a 56-cell ensemble for left and right responses during sound-guided trials. The trajectories were initially indistinguishable and then diverged sharply after the animal made a response. Expanding this analysis to include action blocks revealed population activity patterns that occupied additional, distinct subspaces within the same representational space (Fig. 5b). To quantify rule representations present in the population code, we asked how accurately block type could be predicted from individual population activity vectors. For each session, we constructed a classifier based on linear discriminant analysis (Online Methods). Testing each classifier with fivefold cross-validation revealed that in all cases trial type could be decoded well above chance (Fig. 5c; sound: 78 ± 3\%, action-left: 86 ± 4\%, action-right: 82 ± 3\%; versus chance level of 33\%, P = 1 × 10−6, 1 × 10−6, 5 × 10−7; t8 = 13.2, 12.8, 14.3; one-sample t-test; n = 9 sessions). Repetition of this analysis using a moving window yielded high decoding accuracy at all times during a trial (Fig. 5d), consistent with a global shift in engagement of the network rather than a simple change in the processing of cue, action or outcome related signals. Next we asked whether accuracy of the ensemble classifier could have been driven by a few highly rule-selective cells. When classifiers were trained on ΔF/F of individual cells, we found that 27\% of the cells could be used to decode block types at rates above chance; however, accuracies fell along a continuum and at levels below the accuracy of the ensemble (Fig. 5e). To ensure that the differences in trajectories and decoding accuracies were not due to simple sensory or motor parameters, we computed trajectories with matched stimulus, prior choice, current choice and outcome conditions. Analyses of these congruent trials, which differed only by rule, yielded similar results (Fig. 5f–i and Supplementary Fig. 7a–d). Taken together, these results indicate that the behavioral implementation of specific conditional and nonconditional rules is associated with distinct network activity patterns in M2, such that population activity from any time during behavior can be used to decode task contingencies with high accuracy.

\subsection{Activity toggles between rule-related patterns}
When animals solve trials with the same contingencies a second time, do M2 ensembles revisit similar activity patterns or does population activity migrate to a previously uncharted region of state space? Our task was well suited to address this question because blocks of the same trial type were presented multiple times within the same behavioral session. Figure 6a shows an example set of neural circuit trajectories for the first 12 trial blocks within one behavioral session, in which trajectories could be clearly grouped by block type and not by their temporal order. To quantify the representational similarity of ensemble dynamics on a block-by-block basis, we calculated the mean Euclidean distances between all possible pairwise comparisons of trajectories within an experiment (Online Methods). We found that neural circuit trajectories from blocks of the same type had a relatively small distance of separation and were similarly compact (Fig. 6b and Supplementary Fig. 7e,f; for sound (S), action-left (AL) and action-right (AR): S–S versus AL–AL: P = 0.6, W = 3; S–S versus AR–AR, P = 0.5, W = 13; Wilcoxon signed-rank test). By contrast, trajectories from different block types were represented by markedly different ensemble activity (S–S versus S–AL, P = 0.004, W = 0; S–S versus S–AR, P = 0.004, W = 0; S–S versus AL–AR, P = 0.004, W = 0; corrected α = 0.01, Wilcoxon signed-rank test with Bonferroni correction). These results indicate that, during adaptive decision-making, M2 toggles between distinct functional configurations as the animal repeatedly engages corresponding changes in task demands.

\subsection{Comparison of task-related neural dynamics in M2, ALM and V1}
Next we sought to determine whether the observed neural dynamics are specific to M2 or may be found in other brain regions also. For this purpose, we imaged neural ensembles in layer 2/3 of anterior lateral motor cortex (ALM; 65 ± 6 cells per field of view, mean ± s.e.m.; 8 sessions from 4 mice; Supplementary Fig. 1b) and primary visual cortex (V1; 57 ± 7 cells per field of view; 4 sessions from 2 mice; Supplementary Fig. 1c) to compare with the data from M2 (62 ± 6 cells per field of view; 9 sessions from 5 mice). ALM has been implicated in motor planning and execution37,38; however, it is ∼1.5 mm distant from M2, and the relationship between the two frontal cortical regions is not understood. V1 was chosen as a control region because the task was performed in the dark and involved no visual stimulus. Multiple linear regression analysis showed that M2 neurons robustly encoded not only the choice of the current trial, but also the choices of the two prior trials (Fig. 7a). By contrast, a higher proportion of cells in ALM encoded the current choice; however, the signals decayed faster, resulting in weaker encoding of prior choices (Fig. 7b). Activity of M2 and ALM neurons could prefer either the ipsilateral or contralateral direction (Supplementary Fig. 8a,b), consistent with prior studies30,38. Unexpectedly, choice signals were also observed in V1 (Fig. 7c). Choice selectivity in V1 was relatively weak, and ΔF/F was almost always higher when animals made an ipsilateral choice (Supplementary Fig. 8c). Because choice signals in V1 were transient and animals performed the task in the dark, we conjecture that the selectivity might relate to corollary discharge. To investigate ensemble activity, we employed the same dPCA and linear classifier analyses used for M2. We found that rule type could be decoded with high accuracy using ensemble activity from ALM (matched sound: action-left trials: 78 ± 4\%, t7 = 6.94; P = 2 × 10−4; matched sound: action-right trials: 78 ± 3\%, t7 = 8.68; P = 5 × 10−5; versus chance level of 50\%, one-sample t-test; Fig. 7d), but at a much worse rate for V1 (matched sound: action-left trials: 58 ± 5\%, t3 = 1.88; P = 0.2; matched sound: action-right trials: 67 ± 4\%, t3 = 3.76; P = 0.03). Therefore, both ALM and M2 exhibited task-specific ensemble activity patterns. However, unlike what we found in M2, characterization of ensemble transitions in ALM did not reveal significant differences between switches to sound versus action blocks (sound: xo = 8.2, action: xo = 10.2, median; z = 1.62, P = 0.11; sound: k = 0.37, action: k = 0.56, median; z = 0.89, P = 0.4; Wilcoxon rank-sum test; Fig. 7e). There were also no detectable timing differences between neural and behavioral transitions in ALM (sound: P = 0.5, z = 0.61; action: P = 0.13, z = –1.50; Wilcoxon signed-rank test; neural transition defined as 75\% L). Taken together, these data indicate regionally specific ensemble dynamics associated with adaptive behavior.