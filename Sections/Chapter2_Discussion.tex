
\section{Discussion}
How does the outcome of a chosen action influence how it is represented in the brain? In this study, we used a two-choice discrimination task with probabilistic outcomes to investigate this question in the M2 region of the murine MFC. The results help to illuminate how information related to choices and their outcomes are integrated within the frontal lobe. M2 neurons were found to robustly encode rewarded choices; however, choice-related signals diminished when a rewarding outcome was omitted. Furthermore, an increase in the magnitude of reinforcement had far less impact on choice representations than did its categorical presence or absence. The preferential encoding of rewarded choices in M2 provides a plausible mechanism that may underlie its established role in the learning and implementation of reinforced actions during instrumental behavior.

\paragraph{Cortical Representation of Prior Choices in Rodents}
Optimal performance in the discrimination task required the subject to choose strictly based on auditory cues. In principle, information about past actions could be discarded or ignored. Therefore, in these well-trained animals it was somewhat surprising to observe robust and persistent choice representations, both at the level of single neurons (Figs. \ref{fig:CC_fig4} and \ref{fig:CC_fig5}) and ensembles (Figs \ref{fig:CC_fig6} and \ref{fig:CC_fig7}). However, similar task-irrelevant information coding has been reported elsewhere—in the monkey prefrontal cortex \citep{genovesio2014autonomous}, as well as in the posterior parietal cortex of rodents \citep{morcos2016history}. At the behavioral level, response biases based on choice and outcome history have been observed in human subjects during perceptual tasks even after extensive training \citep{frund2014quantifying, abrahamyan2016adaptable}. Our findings and previous results therefore suggest that under some circumstances, higher-order cortical areas continue to monitor past choices and outcomes, even if task performance does not strictly require such information.

Previous studies have reported that neurons in the rodent MFC encode past choices \citep{sul2011role, siniscalchi2016fast} and their outcomes \citep{kargo2007adaptation, sul2011role, yuan2014cortical}. However, sustained choice-related signals are not unique to this brain region. They have also been found in other nodes of the frontal-striatal network including the dorsomedial striatum \citep{kim2013signals} and orbitofrontal cortex \citep{sul2010distinct}, as well as the posterior parietal cortex \citep{hwang2017history}. This is not to say that all cortical regions exhibit persistent signals associated with chosen actions. For example, our previous study detected only very brief choice signals in the mouse primary visual cortex \citep{siniscalchi2016fast}. Similarly, choice signals like those found in dorsomedial striatum lasted only transiently in dorsolateral striatum \citep{kim2013signals}. This point is notable because M2 and other medial frontal areas send dense projections to dorsomedial striatum, while afferents in dorsolateral striatum come mostly from primary motor cortex \citep{reep1999topographic}.

The extended time course of the choice signals we observed could serve as an eligibility trace that keeps recently performed actions available for learning. In the current study, the representation of choice-related information in M2 neurons persisted into the middle of the next trial (Fig. \ref{fig:CC_fig6}A). It is worth noting that our previous study detected significant choice-related signals over an even longer duration—up to two trials after the corresponding action was chosen \citep{siniscalchi2016fast}. Interestingly, the prior study employed a rule-switching task in which subjects were required to monitor choices and their outcomes. This raises an intriguing possibility: that the temporal scale of choice history signals may depend on the task demands \citep{bernacchia2011reservoir, donahue2013cortical}. Namely, the optimal learning rate depends on the volatility of the environment \citep{behrens2007learning, farashahi2017metaplasticity}. If the persistence of choice representations is indeed a flexible parameter, then it could allow the system to adapt to changes in volatility by serving as a point of adjustment for the temporal integration of choice information.

\paragraph{Enhanced Population Coding for Rewarded Choices}
Our results reveal that neural representations of chosen actions in mouse M2 are outcome-dependent. This finding agrees in principle with a previous study demonstrating that rewarded choices are more reliably encoded relative to unrewarded choices in the primate supplementary eye field and dorsolateral prefrontal cortex during a matching-pennies task \citep{donahue2013cortical}. Interestingly, the effect was not evident in recordings from the same set of neurons during a visual search task in which a visuospatial cue instructed the correct response at the beginning of each trial. The critical difference between these two tasks may be the presence of an instructive cue—a feature which the visual search task shares with the task used in our current study. Notably, we did find a robust reward-associated enhancement of choice coding under these circumstances. One possible explanation is that, in contrast to visuospatial instruction, the cues used in our auditory discrimination task only came to be associated with the correct choice through learning. Therefore, the reward-associated enhancement we observed may reflect an action-monitoring process associated with the maintenance of arbitrary sensorimotor associations—a process which may be unnecessary during the less demanding visual search task. Another possible explanation regards our use of ensemble decoding, which may be more sensitive than decoding from single units and can thus be expected to detect smaller differences between outcome conditions.

In general, the neural correlates of choice and outcome history have been studied using binary outcome conditions in which a reward is either provided or withheld on a given trial. We sought to extend the results of these prior studies by comparing effects of multiple reward magnitudes. Additionally, we were able to dissociate effects of performance and outcome by measuring the impacts of unexpected reward omissions and windfalls in mice trained to a very high level of proficiency (>90% accuracy).

At the behavioral level, reward size affected consummatory licking as well as the likelihood of a response to the next cue, but failed to influence the accuracy of responses (Fig. \ref{fig:CC_fig2}A–C). This suggests that infrequent changes in reward magnitude impacted motivation without significantly influencing choices. Moreover, the neural ensemble representations of chosen actions did not differ between single- and double-reward trials. Instead, the greatest contrast was found between rewarded and unrewarded choices. Our results therefore suggest that the influence of outcome on choice representations in M2 is driven less by the magnitude of reward than by its explicit presence or absence. However, because actions were reinforced immediately in our task design, it remains possible that floor or ceiling effects could have limited the outcome sensitivity of the choice-selective population.

What physiological mechanisms underlie the outcome dependence of choice signals observed in M2? One intriguing possibility concerns the role of neuromodulation, which may directly reconfigure the local network dynamics, or act on inputs to M2. In particular, dopaminergic \citep{schultz1997neural} and cholinergic \citep{hangya2015central} neurons are known to carry signals related to reward. Furthermore, reward-dependent activation of dopaminergic projections to nearby primary motor cortex have been implicated in motor skill learning \citep{hosp2011dopaminergic, leemburg2018motor}. It is therefore interesting to speculate on whether similar mechanisms might contribute to associative learning \citep{takehara2008spontaneous} and more specifically, to the auditory-motor associations necessary for performance of the task presented here. In any case, the impact of neuromodulators on motor cortical choice signaling will comprise an exciting topic for future research.

\paragraph{Persistent Neural and Behavioral Effects Associated with Errors}
The actions chosen on error trials were decoded least accurately from the corresponding ensemble activity (Fig. \ref{fig:CC_fig6}E,F). This result is consistent with an earlier study that revealed disrupted MFC ensemble representations for choices and their outcomes during periods when rats committed multiple errors in a radial arm maze \citep{lapish2008successful, hyman2012action}. The discretized trial structure of our auditory discrimination task allowed us to build upon this prior result by measuring the reliability of ensemble representations into the next trial. Furthermore, the inclusion of omitted-reward trials allowed direct comparisons of ensemble representations associated with correct and incorrect choices, independent of the associated outcomes.

Another previous study demonstrated a tight relationship between sustained error signals in MFC and behavioral performance in the next trial, measured as post-error slowing during a timing task \citep{narayanan2013common}. Similarly, our analysis revealed not only an error-related decrement in the fidelity of neural choice representations, but also a behavioral performance decrement following error trials that could not be explained by reward omission alone. These results may provide some insight into the sources of error for these well-trained subjects. Specifically, the prolonged time course of the neural and behavioral effects associated with errors suggests that they may have arisen in part due to factors that spanned multiple trials—such as periods of hypo- or hyper-arousal. In any case, these results together with prior studies indicate that errors are often associated with persistent internal states that can impact subsequent behavioral performance.

\paragraph{Simultaneous Recording Confers a Modest Decoding Advantage}
Prior theoretical work has demonstrated that correlated variability in neural populations can either degrade or enhance population coding, depending on the interaction between signal and noise correlations \citep{averbeck2003neural, averbeck2006neural}. The analysis shown in Fig. \ref{fig:CC_fig7} revealed that chosen actions could be decoded more accurately from simultaneously recorded ensembles, relative to pseudo-ensembles in which the correlations in neural activity associated with simultaneity (ie noise correlations) had been disrupted. In particular, the observed effect of simultaneity seems to have resulted from the preservation of largely positive correlations in trial-to-trial neural variability unrelated to the chosen action (Fig. \ref{fig:CC_fig7}B,C). Possible sources of noise correlations in our recordings could include unobserved behavior such as whisking, features of the network architecture, or changes in internal state associated with motivation or arousal.

The effect of simultaneity increased with the number of neurons in an ensemble, across the range of ensemble sizes tested (Fig. \ref{fig:CC_fig7}G). Notably, an earlier study in the primate supplementary motor area found no statistically significant effect of correlated spike-count variability on the encoding of movements by ensembles of three to eight neurons \citep{averbeck2006effects}. Our analyses only revealed a consistent simultaneity effect for ensembles larger than nine neurons, which may highlight the utility of large-scale recordings for addressing this question. However, it should be emphasized that even for ensembles of 30 cells, the comparative advantage for ensembles was modest (4\%), and choices could still be decoded from pseudo-ensembles with above chance-level accuracy at every ensemble size. Furthermore, estimated correlations between neurons tend to strengthen at longer timescales \citep{averbeck2003neural}. Hence, the wider time-bins used in our study (500 vs. 66 ms), as well as the slower dynamics associated with calcium imaging could explain why our analyses were more sensitive to correlated variability. We also found that marginal decoding accuracy decreased rapidly as cells were added to the population, for both ensembles and pseudo-ensembles (Fig. \ref{fig:CC_fig7}F,J). This result suggests a high level of redundancy in M2 population codes, similar to previous results found in the rat primary motor cortex during a simple reaction time task \citep{narayanan2005redundancy}.

\paragraph{Insights into the Role of M2 in Goal-Directed Behavior}
The choice selectivity magnitudes of individual neurons (Fig. \ref{fig:CC_fig5}D) and the accuracy of decoding choices from ensemble activity (Fig. \ref{fig:CC_fig6}E,F) both decreased from double- to omitted-rewarded trials, and then further decreased in error trials. How do the observed physiological changes ultimately impact behavior? Causal perturbations aimed at addressing this question will require a more detailed understanding of how genetically \citep{kvitsiani2013distinct, pinto2015cell, kamigaki2017delay} or anatomically identified subtypes of frontal cortical neurons \citep{li2015motor, chen2017map, otis2017prefrontal} contribute to the choice signals observed in our experiments.

Goal-directed behavior requires the capacity to adjust the current policy for action selection according to the impact of past choices on the likelihood of a desired outcome. Our results demonstrate that sustained neural representations of chosen actions in mouse M2 are sensitive to their resultant outcomes, such that rewarded choices are more robustly encoded. In turn, the preferential encoding of rewarded choices could allow the frontal cortex to bias the influence of recent, positively reinforced actions on future decisions. This proposed mechanism would help to explain effects of lesioning \citep{passingham1988premotor, gremel2013premotor} and inactivation \citep{siniscalchi2016fast, makino2017transformation} that have implicated M2 more broadly in the learning and implementation of voluntary behavior. In summary, our results contribute to a growing body of evidence supporting a role for MFC, and M2 more specifically, in the flexible execution of goal-directed actions.