\section{Introduction}

Operant behaviors are structured around stimuli, actions and outcomes. Successful execution of a task requires selecting actions that are consistent with the contingencies between these task variables. Control of action selection in the brain should be both stable and flexible. On the one hand, stability allows a subject to sustain high performance to maximize reward. On the other hand, flexibility is essential for quickly adjusting behavior when a change in contingencies occurs. Striking the delicate balance between stability and flexibility is therefore a key requirement of adaptive decision-making. Moreover, a lack of balance between these opposing aspects of cognitive control is a hallmark of psychiatric disorders \citep{griffiths2014translational}.

How do we know when to be stable or flexible in a changing environment? In tasks without explicit contextual cues, subjects may adjust their response strategy through reward feedback. Prior studies have observed task-dependent differences in neuronal firing rates and selectivity in multiple frontal cortical regions \citep{asaad2000task,rich2009rat,rodgers2014neural}. During periods of behavioral adjustment, evolution of cortical activity was found to be gradual and late, occurring on time courses that generally match or lag the improvement in task performance \citep{mitz1991learning,chen1995neuronal,pasupathy2005different,antzoulatos2011differences}. However, neurons in the frontal cortex exhibit substantial cell-to-cell variability in such time courses \citep{mitz1991learning}. Population activity may therefore be more useful for capturing circuit dynamics \citep{mante2013context,stokes2013dynamic,wills2005attractor}. Using ensemble recordings, two studies examined reward-guided adaptations, and they found the corresponding changes in network activity to be surprisingly abrupt \citep{durstewitz2010abrupt,karlsson2012network}. Determining the functional significance of these findings, however, will require quantitative comparisons of ensemble activity transitions that differ in their dynamics. Transitions that are relatively gradual versus abrupt, or that differ in onset with respect to behavioral changes, could reflect distinct underlying mechanisms for cognitive control.

To study adaptive sensorimotor decision-making in mice, we designed a head-fixed task that required animals to shift many times between three sets of stimulus–response contingencies. This task is a variant of arbitrary sensorimotor mapping, a classic model in which subjects are required to follow conditional rules \citep{bunge2005neural,white1999rule}, such as “for stimulus A, perform one action; for stimulus B, perform another action.” Once learned, the stimulus–response contingencies can then be switched, requiring the learning of novel mappings or retrieval of familiar associations. Associations are made by linking nonspatial stimuli or conditions to actions, and are therefore termed “arbitrary” \citep{wise2000arbitrary}. A number of brain regions are involved in arbitrary sensorimotor mapping, including the frontal lobe, striatum, hippocampus and thalamus \citep{wise2000arbitrary}. Within the frontal lobe, the dorsal premotor cortex has been implicated in the selection of motor programs based on antecedent conditions, as evidenced by the results of lesion studies \citep{petrides1985deficits,halsband1985premotor,nixon2004cortico}, electrophysiology \citep{mitz1991learning,chen1995neuronal}, functional imaging \citep{toni2001learning,boettiger2005frontal} and transcranial stimulation \citep{rushworth2002role} in humans and nonhuman primates.

Secondary motor cortex (M2) has been described as a potential rodent homolog of primate higher-order motor areas \citep{murray2000role,preuss1995rats}. Its location, adjacent to the medial prefrontal and primary motor regions, suggests that it may function as a cognitive–motor interface. A long line of research has linked the premotor cortex and neighboring regions to the generation of volitional movements \citep{nachev2008functional,schall2002monitoring,isoda2007switching}. Recent studies in rodents have also focused on the role of M2 in driving movements. Random-ratio lever-pressing was shown to become insensitive to reward devaluation in M2-lesioned mice, suggesting a role for M2 in goal-directed actions \citep{gremel2013premotor}. Neural activity is modulated before movement, reflecting involvement in action preparation and initiation \citep{sul2011role,erlich2011cortical,murakami2014neural}. Moreover, M2 neurons encode not only current action, but also prior choice and outcome, indicating a broader role in decision-making \citep{sul2011role}. One early study showed that rats with lesions of medial frontal cortex—a broader region that includes M2—had deficits in a visual conditional motor task \citep{passingham1988premotor}.

To elucidate the relationship between frontal ensemble activity and adaptive behavior, we used two-photon calcium imaging to record from M2 neurons in behaving mice. We found distinct population activity patterns associated with each of the three sets of stimulus–response contingencies. Moreover, following a contingency switch, transitions between ensemble patterns occurred earlier and were more abrupt when animals were required to abort repetitive actions and use a conditional rule. In fact, this change in ensemble activity state could be detected after only a few error trials, preceding the more gradual recovery of behavioral performance. Our results uncover distinct neural transitions associated with different phases of voluntary behavior and identify a leading role for M2 in engaging actions that require the use of sensorimotor associations.