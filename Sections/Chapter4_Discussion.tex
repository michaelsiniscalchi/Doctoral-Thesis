\nnsec{Discussion}

We used cell type-specific Ca$^{2+}$ imaging during a rule switching task to study the neural correlates of adaptive sensorimotor behavior in the MFC. Task-related activity was evident in the majority of SST, VIP, PV, and PYR neurons. To measure the reliability of single-unit responses to specific task variables, we calculated a modulation index based on the receiver operating characteristic. Substantial proportions of each cell population carried signals related to choices, outcomes, and the current rule context. These results provide insight on the question of how task representations are distributed among cell types within a cortical region known to function in goal-directed sensorimotor behaviors.

\paragraph{} Based on their subcellular postsynaptic targets, specific classes of GABAergic interneurons are suitably positioned to regulate the flow of activity through cortical networks \citep{kepecs2014interneuron}. Namely, SST+ interneurons preferentially target the dendrites of pyramidal cells, and PV+ interneurons preferentially target their cell body and proximal axon. These subcellular anatomical features correspond to the excitatory inputs and outputs, respectively, of the principal cells which may project within or outside of the local microcircuit. More broadly, SST and PV mediated inhibition, respectively, might serve to route the information flowing into and out of the processing units within MFC. In contrast to other inhibitory cell-types, VIP+ interneurons almost solely target other interneurons, and make particularly strong synapses on SST neurons. Thus, VIP cells appear to specialize in disinhibition \citep{letzkus2011disinhibitory,pi13,karnani2016opening}, and in particular may disinhibit the dendrites of pyramidal neurons through their suppression of SST activity.  

Despite striking differences in synaptic connectivity, all of the inhibitory cell types imaged in our experiments carried signals for the behavioral variables most important for task performance---namely, choices, outcomes, and the rule context governing reinforcement in the current trial. These results suggest that the specialized forms of inhibition that distinguish these populations may all function in the processing of diverse task representations in the MFC. 

Choice signaling was clearly evident in SST, VIP, PV, and PYR neurons regardless of which rule was being enforced, but was more pronounced during action trials. The summation of activity related to current and prior choices may partially account for this difference, due to the repetition of choices demanded by the action rule. We examined the persistence of choice-related signaling during the sound rule, and found that it lasted well into the next trial in all four cell types. We also found limited evidence for preferential activity surrounding contralateral choices during action but not sound blocks---a trend that rose to significance only in VIP populations. Some degree of choice preference would be consistent with previous results of unilateral MFC inactivation, which induced an ipsilateral choice bias during sensorimotor behavior specifically in the absence of instructive spatial cues \citep{erlich2011cortical,erlich2015distinct,hanks2015distinct}. Endogenous differences in the interhemispheric balance of MFC activity may in this case reflect a goal-directed choice bias, which would be one solution to the stimulus-response-outcome contingencies of the action rule.

Trial outcomes were also represented in the activity of all four cell types. In each case, outcome signals persisted throughout the current trial and well into the subsequent trial. SST activity exhibited the strongest and most sustained modulation, with signals remaining notably elevated throughout the intertrial interval. SST populations also included the largest percentage of outcome-responsive neurons. Interestingly, they contained roughly balanced proportions of neurons with preferential activity following rewarded and unrewarded choices. PYR and VIP populations were more heavily recruited following rewarded choices. This result is in agreement with a previous study where reinforcement was found to elicit VIP activity in the primary auditory cortex (A1) during an auditory go/no-go task \citep{pi13}. The same study found that VIP stimulation could indirectly recruit pyramidal neurons in A1 or mPFC through disinhibition.  

Similar to choices and outcomes, the current rule context was also reflected in the activities of all four cell types. In most cases, rule signals were evident for the entire duration of the trial. The proportion of neurons exhibiting differential recruitment across rules was similar among cell types, and differed only during trials where the contralateral spout was chosen. During these trials, a greater proportion of significantly modulated neurons was found in SST as compared to PYR populations. An overall preference for one rule context over the other was found only within the PYR population, which was more heavily recruited during the sound rule as compared to the action rule. We have previously found that bilateral pharmacological inactivation in MFC caused context-dependent effects on behavioral adjustment following a rule switch---namely, adjustment was disrupted under the sound rule and enhanced under the action rule \citep{siniscalchi2016fast}. Preferential PYR activity during the sound rule fits well with idea that MFC should exert greater control over behavior in contexts that require the engagement of arbitrary sensorimotor associations. 

\paragraph{}One important limitation of our approach regards the temporal resolution of the $\frac{\Delta F}{F}$ signals measured in our experiments. The temporal resolution of two-photon Ca$^{2+}$ imaging data is bounded by a number of factors including the sampling rate (which may be dependent on scanner type), the response kinetics of the calcium indicator, and the time course of changes in cytosolic Ca$^{2+}$ concentration caused by action potentials. Scanning methods trade off sampling rate for a greater number of pixels per frame. In order to record from a large number of neurons simultaneously and at cellular resolution, our time-lapse imaging data were obtained at $\sim3.6$ Hz using galvanometric scanners. These data were well-suited to compare signals resulting from changes in firing rates on the scale of hundreds of ms. However, more rapid fluctuations that could be of physiological importance may have been attenuated. Additionally, the intrinsic time course of changes in cytosolic free Ca$^{2+}$ concentration is dependent on subcellular compartmentalization, the compliment of membrane ion channels, calcium release from internal stores, and cytosolic calcium buffers \citep{higley2008calcium,higley2012calcium} all of which can vary by cell type \citep{lee2000differences}.

\paragraph{} During the rule switching task, mice were able to adjust their behavior to meet the abrupt shifts in the contingencies between sensory cues, actions, and outcomes that were associated with transitions between rule contexts.  However, adjustment to the sound rule generally required far fewer trials than the action rule ($39 \pm 3$ vs. $126 \pm 6$; Fig. \ref{fig:Fig3}). This result conflicts with our previous study using an almost identical task, in which subjects required $\sim$ 40 trials to complete sound or action blocks \citep{siniscalchi2016fast}. Two differences in the structure of the earlier version of the task may account for this discrepancy: (1) it included a 500 ms grace period following each sound cue, during which time any recorded responses would not affect the trial outcome, and (2) the intertrial interval (ITI) was set at a constant duration of 7 s between sound cues. In the present study, a choice was made with the first lick following cue onset, and the random ITI (5-16 s, drawn from a truncated exponential distribution with a mean of 8 s) precluded accurate prediction of the next cue onset time. Although unexpected, the results of this manipulation suggest that the ability to inhibit well-learned stimulus-response associations may be disrupted when choices are offered at unpredictable time intervals.    

\paragraph{} Reversible inactivation experiments have implicated the MFC in lateralized decisions informed by either short-term memory \citep{erlich2011cortical,guo2014flow,kopec2015cortical} or the gradual accumulation of sensory evidence \citep{erlich2015distinct,hanks2015distinct}. Additionally, our own recent study found that bilateral MFC inactivation during the rule switching task impaired performance only during adjustment to the sound rule, when the task required the use of learned sensorimotor associations \citep{siniscalchi2016fast}. Using two-photon Ca$^{2+}$ imaging, we also observed transitions in population activity in MFC as the mice adjusted their behavior following a rule switch. Transitions occurred earlier and were more abrupt during the sound rule as compared to the action rule, despite a similar time course of behavioral adjustment. Taken together, these results may suggest that the use of antecedent conditions to guide lateralized action selection comprises a critical function of MFC in decision making.

The local circuit mechanisms within MFC that may realize this cognitive-level function remain unclear. More fundamentally, knowledge of the relationship between afferent and efferent information content may be needed to define the specific computations performed. Further investigations will be necessary to illuminate the precise role of MFC in sensorimotor decision-making---in part by determining how task-related information carried by neuronal activity is transformed within the local network. 