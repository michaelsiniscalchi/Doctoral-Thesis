\nnsec{Introduction}

Adaptive behavior requires the flexible use of information based on learned contingencies in the environment. In particular, sensorimotor decisions benefit from knowledge of the statistical relationships between sensory cues, actions, and rewarding outcomes. However, these relationships can shift with the environmental context as well as internal states such as hunger or satiety. How the brain makes flexible, informed decisions in the face of changing circumstances remains a question of great scientific interest.

In rodents, the medial frontal cortex (MFC) has been implicated in the representation of task variables important for sensorimotor decisions. MFC neurons have been shown to encode information related to recent choices and their outcomes \citep{sul2011role,yuan2015cortical,siniscalchi2019enhanced,mao2019cortical}. One recent study demonstrated that signals for an upcoming choice appear earlier within MFC than in other brain regions including the medial prefrontal cortex (mPFC), motor cortex, and striatum \citep{sul2011role}. Moreover, bilateral inactivation or lesioning of MFC can impair decisions based on prior outcomes \citep{sul2011role} or accumulated sensory evidence \citep{erlich2015distinct}. 

Neural representations of the behavioral context have also been found in MFC \citep{durstewitz2010abrupt,hyman2012contextual,siniscalchi2016fast} as well as adjacent cortical fields in the mPFC \citep{rich2009rat}. In an earlier study, we found that bilateral inactivation of dorsomedial MFC selectively impaired adjustment to sound-guided decisions in a rule switching task that required mice to shift between the use and abandonment of learned auditory-motor associations \citep{siniscalchi2016fast}. 

Collectively, these results suggest that MFC may serve an important role in the selection or planning of specific actions based on sensory cues, prior reinforcement, and behavioral context. However, we lack a detailed understanding of how cortical microcircuits, including those in MFC, process such information during context-dependent decisions. One important basic question regards how this labor might be divided between different cell types of the neocortex. 

Neurons of the cerebral cortex are striking in their diversity, and can be classified based on differences in morphology, physiology, connectivity, and molecular markers \citep{connors1990intrinsic,kubota1994three,kawaguchi1995physiological,tremblay2016gabaergic,huang2019diversity}. However, recently developed mouse lines that selectively express cyclic recombinase (cre) in genetically defined cell types \citep{taniguchi11} have focused intense interest on four non-overlapping cell populations that collectively account for $\sim$ 97\% of cortical neurons \citep{tremblay2016gabaergic}: the somatostatin (SST$^+$), parvalbumin (PV$^+$), and vasointestinal peptide-positive (VIP$^+$) GABAergic interneurons, and the excitatory pyramidal neurons, which express the Ca$^{2+}$-calmodulin dependent kinase CaMKII$\alpha$ (PYR; \citep{jones1994alpha,wang2013distribution}).

Recent studies exploring the role of mPFC cell types in goal-directed behavior have revealed a variety of cell type-specific behavioral correlates, which may suggest distinct roles in the processing of task-related information. The earliest study of this kind \citep{kvitsiani2013distinct} found that entry into a reward zone positioned at the end of a linear track was punctuated by rapid suppression of activity in a sub-population of SST cells, and reward zone exit was marked by a transient increase in PV activity. Subsequent studies have contrasted two or more of the GABAergic populations with regard to their sensitivity to sensory cues, trial outcomes, motor responses, or upcoming choices \citep{pinto2015cell,kim2016distinct}.

We attempted to build on earlier studies focused on task-related signaling by separately examining the activities of four distinct MFC cell types during flexible sensorimotor behavior. Specifically, we used cell type-specific calcium imaging to measure the activity patterns of SST, VIP, PV, and PYR neurons during a two-choice rule switching task, with the goal of estimating the contributions of each cell type to the representation of choices, outcomes, and the rules governing reinforcement. We quantified signals for each of these variables at the single-unit level using a modulation index based on the receiver operating characteristic. The results of these analyses implicate all four cell types in the representation of choices and their outcomes, as well as the reinforcement context in which sensorimotor decisions are made. 