%Running title

\pagestyle{fancy} %For main text, if desired: Switch to fancy header
\fancyhead[L]{CHAPTER \thechapter}
%\fancyfoot{}
%\fancyfoot[C]{\thepage}
%\fancyfoot[CO,RE]{Author Name}


\fancyhead[L]{} 
\fancyhead[R]{\small{SUMMARY AND FUTURE DIRECTIONS}} 

\nnchap{Summary and Future Directions}

\nnsec{Modulation of Choice-Related Signals by Outcome}

Chapter \ref{CC_paper} focused on how populations of neurons in the rodent brain represent and integrate information related to choices and their outcomes during sensorimotor decision making. Specifically, we asked whether the outcome of a chosen action could affect the strength or persistence of its neural representation in MFC. We used two-photon Ca$^{2+}$ imaging to monitor the activity of large ensembles of neurons while mice engaged in a two-choice auditory discrimination task with probabilistic outcomes, and then estimated the impact of choices and outcomes on the activity of single neurons with multiple linear regression. To compare the reliability of population choice signals in different outcome conditions, we trained classifiers to discriminate activity patterns associated with left and right choices, and then tested their decoding accuracy across trials that differed by reward magnitude.

The linear regression analysis revealed sustained modulation of neuronal activity by choices and outcomes that lasted throughout the duration of the current trial and into the next trial. This result agrees with electrophysiological studies in the rat MFC, which have also revealed effects of choices and outcomes on single-unit responses during both value-based \citep{sul2011role} and sensorimotor decisions \citep{yuan2014cortical,mao2019cortical}. Recently, \cite{mao2019cortical} demonstrated that the relative proportions of choice- and outcome-sensitive neurons within MFC can be predictive of learning rate in an operant visuospatial discrimination task.   

At the population level, choices could be decoded at high accuracy throughout the current trial and for several seconds into the next trial. However, choice-related signals diminished in both magnitude and duration when reward was withheld. By contrast, an increase in the magnitude of reinforcement had little impact on choice-related signals. Errors were decoded least accurately from the corresponding ensemble activity. This result is consistent with an earlier study that revealed disrupted MFC ensemble representations for choices and their outcomes during periods when rats committed multiple errors in a radial arm maze \citep{lapish2008successful, hyman2012action}. Taken together, the observed effects of reinforcement on the strength and persistence of choice-related signals could comprise a mechanism to keep such representations available for learning, or for adjustment of action selection policy in changing environments.

What physiological mechanisms might underlie the outcome dependence of choice signals observed in MFC? One intriguing possibility concerns the role of neuromodulation, which may directly reconfigure the local network dynamics or act on its inputs. In particular, dopaminergic \citep{schultz1997neural} and cholinergic \citep{hangya2015central} neurons are known to carry signals related to reward. Furthermore, reward-dependent activation of dopaminergic projections to nearby primary motor cortex have been implicated in motor skill learning \citep{hosp2011dopaminergic, leemburg2018motor}. It is therefore interesting to speculate on whether similar mechanisms might contribute to associative learning, and more specifically, to the auditory-motor associations necessary for performance of the task presented here. In any case, the impact of neuromodulators on cortical choice signaling will comprise an exciting topic for future research.

The results in Chapter \ref{CC_paper} may provide insights into the associative mechanisms underlying goal-directed action selection. Neural representations of chosen actions in mouse MFC were sensitive to their resultant outcomes, such that rewarded choices were more robustly encoded. Preferential encoding of rewarded choices could promote the influence of recent, positively reinforced actions on future decisions. This proposed mechanism would help to explain effects of lesioning \citep{passingham1988premotor, gremel2013premotor} and inactivation \citep{siniscalchi2016fast, makino2017transformation} that have implicated MFC more broadly in the learning and implementation of voluntary behavior.

\nnsec{Ensemble Dynamics associated with Contextual Shifts during Goal-Directed Behavior}

In Chapter \ref{NN_paper} we examined the neural dynamics associated with adaptive sensorimotor decision-making in mice. We modified the basic task shown in Figure \ref{fig:Intro_ExpSetup} to include three distinct stimulus-response mappings (rules), and required subjects to shift among these rules multiple times within a single session. We imaged populations of MFC neurons while mice participated in this novel rule switching task in order to determine whether the distinct rule contexts governing action selection would be encoded in the population activity patterns---and if so, whether transitions between the associated contextual representations could differ in their dynamics. To address the causal relationship between MFC activity and behavioral adjustment, in a separate set of experiments we bilaterally inactivated MFC during the task using the GABA$_\text{A}$-receptor agonist muscimol.  

Previous studies have reported neural activity changes within multiple frontal cortical regions following a shift in task contingencies, both in single units \citep{asaad2000task,rich2009rat,rodgers2014neural} and at the population level \citep{pasupathy2005different,rich2009rat,antzoulatos2011differences,mante2013context,stokes2013dynamic}. In the rodent MFC, population activity dynamics associated with adjustment to changing task contingencies have been found to be surprisingly abrupt \citep{durstewitz2010abrupt,karlsson2012network}. However, the rate of neural transition dynamics may be difficult to interpret without quantitative comparisons between regions \citep{pasupathy2005different,antzoulatos2011differences} or potentially, between behavioral contexts. Network transitions that differ in their relative rates of change, or in their onset with respect to behavioral changes, could reflect distinct underlying mechanisms for adaptive behavioral control.

Our analysis revealed distinct population activity patterns associated with each of the three rules. Moreover, the transitions between activity patterns occurred earlier and were more abrupt during adjustment to the sound rule---when subjects were required to engage learned sensorimotor associations---than during adjustment to the action rule, when subjects were required to disregard these associations. In fact, changes in ensemble activity state could be detected after only about five trials governed by the sound rule, preceding the more gradual recovery of behavioral performance.  

Effects of inactivation were context dependent: adjustment was disrupted under the sound rule, and surprisingly, was enhanced under the action rule. A unified explanation for these asymmetric effects on behavior is that under these task conditions, MFC activity normally biases a conditional action selection strategy guided by sensory cues. Thus, MFC inactivation may remove a brake on unconditional strategies. Importantly, the manipulation slowed adjustment to the sound rule but did not preclude the eventual transition to high performance. Therefore, the MFC may facilitate the use of a conditional strategy, but is not strictly necessary. 

Together, these results indicate that behavioral adaptation can be associated with distinct neural transition dynamics that depend on the specific contingencies enforced by the environment and the resulting constraints on behavioral strategy. In particular, the requirement to repeatedly engage and abandon a conditional response strategy based on instructive sensory cues provided useful insights on the function of rodent MFC in flexible sensorimotor behavior. Our results suggest that MFC activity may promote the selection of actions based on arbitrary associations between sensory cues and actions, which is reminiscent of functions attributed to higher-order motor areas of the primate brain \citep{mitz1991learning,chen1995neuronal,wise2000arbitrary}.

\nnsec{Cell Types in Task-Related Information Processing}

In Chapter \ref{CellTypes_paper}, we examined the activities of four distinct MFC cell types during flexible sensorimotor behavior. Specifically, we used cell type-specific Ca$^{2+}$ imaging to measure the activity patterns of SST, VIP, PV, and PYR neurons while subjects engaged in the rule-switching task described in Chapter \ref{NN_paper}. Our goal was to estimate the contributions of each cell type to the representation of choices, outcomes, and the rules governing reinforcement. We quantified signals for each of these variables at the single-unit level using a modulation index based on the receiver operating characteristic. Task-related activity was evident in the majority of SST, VIP, PV, and PYR neurons. Furthermore, substantial proportions of each cell population carried signals related to choices, outcomes, and the current rule context. These results may provide insight on the question of how task representations are distributed among cell types within a cortical region known to function in goal-directed sensorimotor behaviors.

Based on their subcellular postsynaptic targets, it has been suggested that specific classes of GABAergic interneurons may be critical in regulating the flow of activity through cortical networks \citep{kepecs2014interneuron}. For example, SST interneurons preferentially target the dendrites of pyramidal cells, and PV interneurons preferentially target their cell body and proximal axon. These subcellular anatomical features correspond to the excitatory inputs and outputs, respectively, of the principal cells which may project within or outside of the local microcircuit. More broadly, SST and PV mediated inhibition, respectively, might serve to route the information flowing into and out of the processing units within MFC. In contrast to other inhibitory cell-types, VIP+ interneurons almost solely target other interneurons, and make particularly strong synapses on SST neurons. Thus, VIP cells appear to specialize in disinhibition \citep{letzkus2011disinhibitory,pi13,karnani2016opening}, and in particular may disinhibit the dendrites of pyramidal neurons through their suppression of SST activity.  

Despite striking differences in synaptic connectivity among inhibitory cell types \citep{kepecs2014interneuron}, all of the cell types imaged in our experiments carried signals for the behavioral variables most important for task performance---namely, choices, outcomes, and the rule context governing reinforcement in the current trial. Choice signaling was clearly evident in SST, VIP, PV, and PYR neurons regardless of which rule was being enforced, and persisted well into the next trial in all four cell types. 

Likewise, trial outcomes were represented in the activity of all four cell types, persisting throughout the current trial and well into the subsequent trial. SST populations exhibited the strongest and most sustained modulation, and also included the largest percentage of outcome-responsive neurons. PYR and VIP populations were preferential recruited following rewarded choices---consistent with a previous study suggesting that reinforcement may indirectly activate PYR neurons through VIP-mediated disinhibition \citep{pi13}.  

The current rule context was also reflected in the activities of all four cell types. Rule signals were evident for the entire duration of the trial, and the proportion of neurons exhibiting differential recruitment across rules was similar among cell types. An overall preference for one rule context over the other was found only within the PYR population, which was more heavily recruited during the sound rule as compared to the action rule. In Chapter \ref{NN_paper}, we found that bilateral pharmacological inactivation in MFC caused context-dependent effects on behavioral adjustment following a rule switch---namely, adjustment was disrupted under the sound rule and enhanced under the action rule \citep{siniscalchi2016fast}. Preferential PYR activity during the sound rule accords with the idea that MFC should exert greater control over behavior in contexts that require the engagement of arbitrary sensorimotor associations. 

Taken together, these results suggest that the specialized forms of inhibition that distinguish SST, PV, and VIP populations may all function in the processing of diverse task representations in the MFC.  

\nnsec{Future Directions}

Recent studies employing reversible inactivation have implicated the MFC in lateralized decisions informed by either short-term memory \citep{erlich2011cortical,guo2014flow,kopec2015cortical} or the gradual accumulation of sensory evidence \citep{erlich2015distinct,hanks2015distinct}. Additionally, our own research points to a role for MFC in the flexible engagement of sensorimotor associations during a left-or-right licking task (Chapter 2 and \cite{siniscalchi2016fast}). Taken together, these results may suggest that the use of antecedent conditions to guide lateralized action selection comprises a critical function of MFC in decision making.

The local circuit mechanisms within MFC that may realize this cognitive-level function remain unclear. More fundamentally, knowledge of the relationship between afferent and efferent information content may be needed to define the specific computations performed. Systematic optogenetic inactivation of the main input and output connections of MFC during conditional sensorimotor tasks could help determine which of these can causally impact behavior at different time points within a trial. If causally relevant inputs and outputs can be identified, then simultaneous electrophysiological recordings across brain areas, and across layers of MFC could help illuminate how the information carried by neuronal activity is transformed within the local network.  

In particular, the orbitofrontal cortex (OFC), which provides dense anatomical input to MFC \citep{reep1984afferent,reep1999topographic}, would be an intriguing candidate to explore. OFC has been implicated in the prediction of outcomes based on sensory cues \citep{ostlund2009evidence,schoenbaum2011does}, as well as in computations incorporating the value of chosen actions \citep{sul2010distinct}. MFC also receives direct input from primary sensory cortices associated with vision, hearing, and touch \citep{reep1999topographic,barthas2017secondary}. Based on the results of inactivation during the rule-switching task (Chapter \ref{NN_paper}), one intriguing possibility is that sensory-related activity might propagate through the network differentially in accordance with task demands. Candidate outputs that could mediate impacts of the MFC on behavior include axon terminal fields within the primary motor cortex and superior colliculus, both of which have been directly implicated in motor control \citep{donoghue1982motor,gandhi2011motor}.   
