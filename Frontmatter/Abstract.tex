\begin{center}
\begin{singlespace}

\textbf{Abstract}
\bigskip

Representations of Choices, Outcomes, and Context\\
in the Medial Frontal Cortex of the Mouse\\
\bigskip

Michael James Siniscalchi\\
\bigskip
2020\\
\bigskip

\end{singlespace}
\end{center}

Adaptive behavior requires the flexible use of information to guide actions based on learned contingencies in the environment. In particular, sensorimotor decisions benefit from knowledge of the statistical relationships between sensory cues, actions, and rewarding outcomes. Sensory cues can become instructive when they consistently predict which actions will lead to a favorable outcome. However, these relationships can shift with the environmental context as well as internal states such as hunger or satiety. To select appropriate actions in the face of changing circumstances, the brain must integrate information related to sensory cues as well as past experience and context. How this is accomplished mechanistically remains a question of great scientific interest.

This dissertation will present three independent studies focused on the representation of task-relevant information by cerebral cortical neurons during sensorimotor decision making. Each of these studies takes a similar reductionist experimental approach to examine the neurophysiological correlates of sensorimotor behavior, using the medial frontal cortex (MFC) of the mouse as a model system. In each case, we make use of a similar two-choice sensorimotor decision-making task that can be performed by head-restrained mice under the objective of a two-photon fluorescence microscope used for simultaneously measuring neural activity.

In Chapter \ref{CC_paper}, we explore the associative mechanisms underlying goal-directed action selection, focusing on the question of how populations of neurons in the MFC may represent and integrate information related to choices and their outcomes. One intriguing hypothesis is that a choice’s outcome could affect the strength or persistence of its neural signature in the MFC. To test this idea, we trained mice on a two-choice auditory discrimination task and then introduced a probabilistic reinforcement schedule during the two-photon imaging sessions. We applied linear discriminant analysis to the population activity patterns in order to estimate the reliability of choice signals under different outcome conditions, and found that rewarding outcomes boosted the fidelity and persistence of choice signals encoded in the population activity. 
% Our results suggest one plausible cortical mechanism for the reinforcement of rewarded actions. Namely, rewarding outcomes can lead to a more robust population-level read-out of recent choice history.

In Chapter \ref{NN_paper}, we examine the population activity dynamics associated with adaptive sensorimotor behavior. Here, we imaged neural activity in the MFC while mice engaged in a novel rule switching task that required them to shift periodically among three distinct stimulus-response-outcome mappings (rules). We found that engagement with each rule was reflected in distinct population activity states, and that furthermore, transitions between states could differ in their dynamics depending on the constraints placed on action selection by the current rule. Specifically, during adjustment to a conditional rule (ie, $\mathit{Stimulus}_A \to \mathit{Response}_C$; $\mathit{Stimulus}_B \to \mathit{Response}_D$), the transition occurred earlier and was more rapid than for an unconditional rule (ie, $\mathit{Stimulus}_A$ $\mathit{or}$ $\mathit{Stimulus}_B$ $\to \mathit{Response}_C$, for only two stimuli). To address the causal relationship between MFC activity and behavioral adjustment to conditional and unconditional rules, we pharmacologically inactivated MFC during the task, and found that the manipulation selectively impaired adjustment to the conditional rule.

In Chapter \ref{CellTypes_paper}, we characterize the contributions of four distinct neocortical cell types to the representation of task-related information in the MFC. We separately measured the activity of SST, VIP, PV, and pyramidal neurons while subjects engaged in the rule switching task from Chapter \ref{NN_paper}, and quantified task-related activity using a modulation index based on the receiver operating characteristic. Substantial proportions of each cell population carried signals related to choices, outcomes, and the rules governing reinforcement.
